%!TEX root = ../template.tex
%%%%%%%%%%%%%%%%%%%%%%%%%%%%%%%%%%%%%%%%%%%%%%%%%%%%%%%%%%%%%%%%%%%%
%% chapter4.tex
%% NOVA thesis document file
%%
%% Chapter with lots of dummy text
%%%%%%%%%%%%%%%%%%%%%%%%%%%%%%%%%%%%%%%%%%%%%%%%%%%%%%%%%%%%%%%%%%%%
\chapter{Implementation of iCBD-Replication and Cache Server}
\label{cha:replication}

This chapter addresses the implementation of the central topics of this thesis, divided into two major fields.
The first section talks about the creation of a middleware system that provides replication features in an integrated way to the iCBD platform. A detailed description of the concept and architectural model, as well as the implementation decisions, can be found in this section.
Then, we show how the performance of the platform clients can be heightened by setting up a client-side caching system that stores images adjacent to the consumers. Concluding in exploring the challenges of recreating the complete platform in a new environment and implementing a real-world scenario at Nova University Computer Science department laboratories.

%%-------------------------------------------------------------------
%%	4 - Implementation of a Replication Module
%%-------------------------------------------------------------------
\section{Implementation of a Replication Module}
\label{sec:replication_impl}


%%-------------------------------------------------------------------
%%	4. - Requirements of the Module
%%-------------------------------------------------------------------
\subsection{Requirements of the Module}
\label{sub:requirements_icbdrep}


%%-------------------------------------------------------------------
%%	4. - Preliminary tests on the BTRFS Incremental Backup features
%%-------------------------------------------------------------------
\subsubsection{Preliminary tests on the BTRFS Incremental Backup features}
\label{subsub:pre_test_btrfs}

%https://btrfs.wiki.kernel.org/index.php/Incremental_Backup
%https://www.samba.org/ftp/rsync/rsync.html

Before the start of any implementation, there was the need to validate the capabilities of the BTRFS file system regarding sending snapshots across different systems.
As one of the requirements was the efficiency of the transference of data. So a small comparison was in order. 


%%-------------------------------------------------------------------
%%	4. - Image Repository
%%-------------------------------------------------------------------
\subsection{Image Repository}
\label{sub:rep_image_repo}

\subsubsection{iCBD Snapshot Structure}
\label{subsub:icbd_snapshot}

In the replication module we treat a snapshot not only as raw data, but a collection of data and metadata that is essencial to unequivocally distinguish the multiple images present in the system. 


%%-------------------------------------------------------------------
%%	4. - Communications between image repositories
%%-------------------------------------------------------------------
\subsection{Communications between image repositories}
\label{sub:rep_rpcs}


%%-------------------------------------------------------------------
%%	4. - Pyro4 Library
%%-------------------------------------------------------------------
\subsubsection{Pyro4 Library}
\label{subsub:rep_pyro4}

%https://pythonhosted.org/Pyro4/


%%-------------------------------------------------------------------
%%	4. - Master Node
%%-------------------------------------------------------------------
\subsection{Master Node}
\label{sub:rep_master_node}


%%-------------------------------------------------------------------
%%	4. - Replica Node
%%-------------------------------------------------------------------
\subsection{Replica Node}
\label{sub:rep_replica_node}






%%-------------------------------------------------------------------
%%	4. - Building a iCBD Cache Server
%%-------------------------------------------------------------------
\section{Building a iCBD Cache Server}
\label{sec:cache_server}

Found a Centos 7 kernel bug.
%https://bugs.centos.org/view.php?id=14228
%https://bugzilla.redhat.com/show_bug.cgi


%%-------------------------------------------------------------------
%%	4. - Services
%%-------------------------------------------------------------------
\subsection{Services}
\label{sub:cache_services}


%%-------------------------------------------------------------------
%%	4. - Networking
%%-------------------------------------------------------------------
\subsection{Networking}
\label{sub:cache_networking}


