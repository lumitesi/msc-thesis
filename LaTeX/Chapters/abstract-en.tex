%!TEX root = ../template.tex
%%%%%%%%%%%%%%%%%%%%%%%%%%%%%%%%%%%%%%%%%%%%%%%%%%%%%%%%%%%%%%%%%%%%
%% abstrac-en.tex
%% NOVA thesis document file
%%
%% Abstract in English
%%%%%%%%%%%%%%%%%%%%%%%%%%%%%%%%%%%%%%%%%%%%%%%%%%%%%%%%%%%%%%%%%%%%

Over the span of a few years, there were fundamental changes in the way computing power is handled. The heightening of virtualisation changed the infrastructure model of a \textit{data centre} and the way physical computers are managed. This shift is the result of allowing for fast deployment of \glspl{VM} in a high consolidation ratio environment and with minimal need for management.

New approaches to virtualisation techniques are being developed at a never seen rate. Which leads to an exciting and vibrating ecosystem of platforms and services seeing the light of day. We see big industry players engaging in such problems as \textit{Desktop Virtualisation} with moderate success, but completely ignoring the already present computation power in their clients, instead, opting for a costly solution of acquiring powerful new machines and software. There is still space for improvement and the development of technologies that take advantage of the onsite computation capabilities with minimum effort on the configuration side.

This thesis focuses on the development of mechanisms for the replication and caching of \textit{VM} images stored in a conventional file system with the ability to perform snapshots. There are some particular items to address: like the solution needs to follow an entirely distributed architecture and fully integrate with a parallel implemented client-based \gls{VDI} platform; needs to work with very large read-only files some of them resulting from the creation of snapshots while maintaining some versioning features. This work will also explore the challenges and advantages of deploying such system in a high throughput network, maintaining high availability and scalability properties while supporting a broad set of clients efficiently. 


% Palavras-chave do resumo em Inglês
%\begin{keywords}
%	\ldots
%\end{keywords} 
