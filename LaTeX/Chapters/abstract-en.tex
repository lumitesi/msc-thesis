%!TEX root = ../template.tex
%%%%%%%%%%%%%%%%%%%%%%%%%%%%%%%%%%%%%%%%%%%%%%%%%%%%%%%%%%%%%%%%%%%%
%% abstrac-en.tex
%% NOVA thesis document file
%%
%% Abstract in English
%%%%%%%%%%%%%%%%%%%%%%%%%%%%%%%%%%%%%%%%%%%%%%%%%%%%%%%%%%%%%%%%%%%%

Recently, in a relatively short timeframe, there were fundamental changes in the way computing power is used. Virtualisation technology has changed both the model of a data centre’s infrastructure and the way physical computers are now managed. This shift is a consequence of today’s fast deployment rate of Virtual Machines (VM) in a high consolidation environment with minimal need for human management. 

New approaches to virtualisation techniques are being developed at a surprisingly fast rate, leading to a new exciting and vibrating ecosystem of platforms and services. We see the big industry players tackling problems such as \textit{Desktop Virtualisation} with moderate success, but completely ignoring the computation power already present in their clients’ infrastructures and, instead, opting for a costly solution based on powerful new machines. There’s still room for improvement in VDI and development of new architectures that take advantage of the computation power available at the user’s desk, with a minimum effort on the management side; iCBD is one of these projects. 

This thesis focuses on the development of mechanisms for the replication and caching of \textit{VM} images stored in a local filesystem, albeit one with the ability to perform snapshots. In this work, there are some challenges to address: the proposed architecture must be entirely distributed and completely integrated with the already existing client-based Virtual Desktop Infrastructure (VDI) platform; and it must be able to efficiently cope with very large, read-only files, (some of them snapshots) and handle their multiple versions. This work will also explore the challenges and advantages of deploying such a system in a high throughput network, with both high availability and scalability while efficiently supporting a large number of users (and their workstations).


% Palavras-chave do resumo em Inglês
\begin{keywords}
	VDI, BTRFS, Snapshots, Replication Middleware, Cache Servers.
\end{keywords} 
