%!TEX root = ../template.tex
%%%%%%%%%%%%%%%%%%%%%%%%%%%%%%%%%%%%%%%%%%%%%%%%%%%%%%%%%%%%%%%%%%%%
%% abstrac-pt.tex
%% NOVA thesis document file
%%
%% Abstract in Portuguese
%%%%%%%%%%%%%%%%%%%%%%%%%%%%%%%%%%%%%%%%%%%%%%%%%%%%%%%%%%%%%%%%%%%%

Nos últimos anos, tem-se assistido a mudanças fundamentais na forma como a capacidade computacional é utilisada. Com o grande aumento da utilização da virtualização a forma como são geridas as máquinas físicas e os modelos de infraestruturas num centro de dados sofreram grandes alterações. Esta mudança é o resultado de uma procura por uma forma de disponibilizar rapidamente uma VM num ambiente altamente consolidado e com a mínima necessidade de intervenção para a sua gestão.

Estão a ser desenvolvidas novas abordagens às técnicas de virtualização a um ritmo nunca visto. O que leva à existência de um ecossistema altamente volátil com novas plataformas e serviços a serem criados a todo o momento. É possível apreciar a entrega de grandes empresas da industria das tecnologias de informação a problemas como a virtualização de desktops com algum sucesso, mas ignorando completamente o poder de computação que já está presente nos seus clientes. Optando ao em vez, por uma via de alto custo, adquirindo máquinas poderosas e vários software. Existe ainda espaço para melhores soluções e para o desenvolvimento de tecnologias que façam uso das capacidades de computação já se encontrem presentes com o mínimo de esforço na sua configuração.

Esta tese foca-se no desenvolvimento de mecanismos de replicação e caching para imagens de maquinas virtuais armazenadas num sistema de ficheiros convencional com a funcionalidade de fazer snapshots. Existem alguns pontos em particular a endereçar: a solução tem que seguir uma arquitectura distribuída e ser totalmente integrada numa solução client-based VDI; tem que funcionar com enormes ficheiros apenas de leitura alguns deles resultantes da criação de snapshots mantendo a característica de manutenção de versões. Este trabalho também incide na exploração dos benefícios de utilizar tal sistema numa rede com uma alta taxa de transferência de dados, em quanto mantem propriedades de alta disponibilidade e escalabilidade suportando um largo conjunto de clientes de forma eficiente.


% Palavras-chave do resumo em Português
%\begin{keywords}
%Palavras-chave (em Português) \ldots
%\end{keywords}
% to add an extra black line
