%!TEX root = ../template.tex
%%%%%%%%%%%%%%%%%%%%%%%%%%%%%%%%%%%%%%%%%%%%%%%%%%%%%%%%%%%%%%%%%%%%
%% chapter6.tex
%% NOVA thesis document file
%%
%% Chapter with lots of dummy text
%%%%%%%%%%%%%%%%%%%%%%%%%%%%%%%%%%%%%%%%%%%%%%%%%%%%%%%%%%%%%%%%%%%%
\chapter{Conclusions \& Future Work}
\label{cha:conclusion}

\section{Conclusions}
\label{sec:con_conclusions}

\textbf{TOPICS :}
\begin{itemize}
	\item The requirements of the implementation of the iCBD-Replication were achieved 
	\item Simple replication of iMI though multiple nodes with a subscription model, where the node only receives the iMIs that really want.
	\item Successful creation of a complete iCBD platform in FCT NOVA campus.
	\item Verifica-se que a introdução do cache server aumenta um pouco o tempo de boot dos clientes, mas mais importante verifica-se capaz de aguentar o boot simultâneo de um Lab. completo ficando pelos 20\% de ocupação de CPU. Verificando-se assim uma alta escalabilidade horizontal da solução.
	\item Two department lab running the solution as a trial, and instrumental in getting experimental results
\end{itemize}

%As a result of our research and experimentation, we found that:

%\begin{itemize}
%	\item bla..
%\end{itemize}

\section{Future Work}
\label{sec:con_future_work}

\textbf{TOPICS :}
\begin{itemize}
	\item Replica to Replica iMI transfer
	\item Diskless Servers / selfhosting - provision iCBD Servers from iMI
	\item Micro Services, as started with this thesis (building iCBD-Replication as a standalone application) the functionalities of the iCBD Core can be segmented in multiple small services, in order to achieve a better use of resources and an easier deployment in a multi homed scenario.
	\item Infrastructure as Code, orchestrate and automate all the process described in the chapter Implementation of a Cache Server.
\end{itemize}




