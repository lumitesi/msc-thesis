%!TEX root = ../template.tex
%%%%%%%%%%%%%%%%%%%%%%%%%%%%%%%%%%%%%%%%%%%%%%%%%%%%%%%%%%%%%%%%%%%%
%% chapter3.tex
%% NOVA thesis document file
%%
%% Chapter with iCBD project
%%%%%%%%%%%%%%%%%%%%%%%%%%%%%%%%%%%%%%%%%%%%%%%%%%%%%%%%%%%%%%%%%%%%


%%-------------------------------------------------------------------
%%	3 - iCBD - Infrastructure for Client-Based (Virtual) Desktop (Computing)
%%-------------------------------------------------------------------
\chapter{iCBD - Infrastructure for Client-Based Desktop}
\label{cha:icbd}

The acronym \gls{iCBD} stands for Infrastructure for Client-Based (Virtual) Desktop (Computing). Is a platform being developed by an R\&D partnership between \textit{NOVA LINCS}, the Computer Science research unit hosted at the \textit{Departamento de Informática of Faculdade de Ciências e Tecnologia of Universidade NOVA de Lisboa} (DI-FCT NOVA) and \textit{SolidNetworks – Business Consulting, LDA} part of the \textit{Reditus S.A.} group. 

Where the primary goal is to achieve a particular kind of \gls{VDI} infrastructure, a client based VDI, where client's computations are performed directly on the client hardware opposed to on big and expensive servers.

This chapter will address the central concepts and associated technologies encompassed in this project, particularly:\\

\begin{description}
	%
	\item [Section~\ref{sec:icbd_concept}] overviews the core concepts of the project and particularly note the limitations and peculiarities of current implementations in contrast with the chosen approach.
	%
	\item [Section~\ref{sec:icbd_architecture}] studies the principal architectural components of the platform, with emphasis on the different layers and how  they act together to serve the end-user.
	%
\end{description}


%%-------------------------------------------------------------------
%%	3.1 - The Concept
%%-------------------------------------------------------------------
\section{The Concept} % (fold)
\label{sec:icbd_concept}

The iCBD as a project pretends to investigate and develop an architecture that leads to the birth of a platform that can operate desktop virtualisation (\gls{VDI}). In a sense, the goal is similar to a client-based VDI, but with the distinction of maintaining all the benefits of both client-based and server-based VDI. Additionally, it should present the power of working as a Cloud \gls{DaaS} without any of the bad traits of the approaches as mentioned earlier.

The aim is to preserve the convenience and simplicity of a fully centralised management platform for Linux and Windows desktops, instantiating those in each physical workstation from virtual machine templates (VMs) kept in repositories. We will talk more about this subject in section \ref{sec:icbd_architecture}\\

To summarise the platform should be able to:

\begin{itemize}
	\item Tuning to a wide range of server configurations, without prejudice to the defined architecture.
	%
	\item Minimize disruption in the use of workstations for end-users. Offering a work environment and experience of use so close to the traditional one that they should not be able to tell from a standard local installation of an \gls{OS} to the use of this platform.
	%
	\item Simplify installation, maintenance and platform management tasks for the entire infrastructure, including servers in their multiple roles, storage and network devices from a single point.
	%
	\item Allow for a highly competitive per workstation cost.
	%
	\item Maintain an inter-site solution; such a geographically disperse multi-office structure.
\end{itemize}


%%-------------------------------------------------------------------
%%	3.2 - The Architecture 
%%-------------------------------------------------------------------
\section{The Architecture} % (fold)
\label{sec:icbd_architecture}


%%-------------------------------------------------------------------
%%	3.2. - Boot Layer 
%%-------------------------------------------------------------------
\subsection{Boot Layer}
\label{sub:icbd_architecture_boot}


%%-------------------------------------------------------------------
%%	3.2. - Administration Layer 
%%-------------------------------------------------------------------
\subsection{Administration Layer}
\label{sub:icbd_architecture_adm}


%%-------------------------------------------------------------------
%%	3.2. - Client Layer 
%%-------------------------------------------------------------------
\subsection{Client Layer}
\label{sub:icbd_architecture_client}

%%-------------------------------------------------------------------
%%	3.2. - Storage Layer 
%%-------------------------------------------------------------------
\subsection{Storage Layer}
\label{sub:icbd_architecture_storage}


%%-------------------------------------------------------------------
%%	3.2. - Replication and Caching - The Problem 
%%-------------------------------------------------------------------
\section{Replication and Caching - The Problem}
\label{sec:replication_cache}


\subsection{Motivation and Goals}
\label{sub:motivation_goals}

\subsection{System Overview}
\label{sub:system_overview}

\subsection{Requirements}
\label{sub:requirements}