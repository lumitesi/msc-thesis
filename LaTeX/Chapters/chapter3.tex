%!TEX root = ../template.tex
%%%%%%%%%%%%%%%%%%%%%%%%%%%%%%%%%%%%%%%%%%%%%%%%%%%%%%%%%%%%%%%%%%%%
%% chapter3.tex
%% NOVA thesis document file
%%
%% Chapter with iCBD project
%%%%%%%%%%%%%%%%%%%%%%%%%%%%%%%%%%%%%%%%%%%%%%%%%%%%%%%%%%%%%%%%%%%%


%%-------------------------------------------------------------------
%%	3 - iCBD - Infrastructure for Client-Based (Virtual) Desktop (Computing)
%%-------------------------------------------------------------------
\chapter{iCBD - Infrastructure for Client-Based Desktop}
\label{cha:icbd}

The acronym \gls{iCBD} stands for Infrastructure for Client-Based (Virtual) Desktop (Computing). Is a platform being developed by an R\&D partnership between \textit{NOVA LINCS}, the Computer Science research unit hosted at the \textit{Departamento de Informática of Faculdade de Ciências e Tecnologia of Universidade NOVA de Lisboa} (DI-FCT NOVA) and \textit{SolidNetworks – Business Consulting, LDA} part of the \textit{Reditus S.A.} group. 

Where the primary goal is to achieve a particular kind of \gls{VDI} infrastructure, a client based VDI, where client's computations are performed directly on the client hardware as opposed to on big and expensive servers. With the distinctive characteristic of not having the necessity of investing in hard disks for the client devices, as well as hoping to solve prominent predicaments in the administration and management of large-scale computer infrastructure.

This chapter will address the central concepts and associated technologies encompassed in this project, particularly:\\

\begin{description}
	%
	\item [Section~\ref{sec:icbd_concept}] overviews the core concepts of the project and particularly note the limitations and peculiarities of current implementations in contrast with the chosen approach.
	%
	\item [Section~\ref{sec:icbd_architecture}] studies the principal architectural components of the platform, with emphasis on the different layers and how  they act together to serve the end-user.
	%
	\item [Section~\ref{sec:replication_cache}] will at last state the problem and motivation for introduction replication techniques in the storage components of the platform. Moreover, the section prefaces the importance of the implementation of cache servers that hold part of the distribution burden and crucial for the support of an increased number of clients.
\end{description}


%%-------------------------------------------------------------------
%%	3.1 - The Concept
%%-------------------------------------------------------------------
\section{The Concept} % (fold)
\label{sec:icbd_concept}

The iCBD as a project pretends to investigate and develop an architecture that leads to the birth of a platform that can operate desktop virtualisation (\gls{VDI}). In a sense, the goal is similar to a client-based VDI, but with the distinction of maintaining all the benefits of both client-based and server-based VDI. Additionally, it should present the power of working as a Cloud \gls{DaaS} without any of the bad traits of the approaches as mentioned earlier.

The aim is to preserve the convenience and simplicity of a fully centralised management platform for Linux and Windows desktops, instantiating those in each physical workstation from virtual machine templates (VMs) kept in repositories. We will talk more about this subject in section \ref{sec:icbd_architecture}\\

To summarise the platform should be able to:

\begin{itemize}
	\item Tuning to a wide range of server configurations, without prejudice to the defined architecture.
	%
	\item Minimize disruption in the use of workstations for end-users. Offering a work environment and experience of use so close to the traditional one that they should not be able to tell from a standard local installation of an \gls{OS} to the use of this platform.
	%
	\item Simplify installation, maintenance and platform management tasks for the entire infrastructure, including servers in their multiple roles, storage and network devices from a single point.
	%
	\item Allow for a highly competitive per workstation cost.
	%
	\item Maintain an inter-site solution; such a geographically disperse multi-office structure.
\end{itemize}


%%-------------------------------------------------------------------
%%	3.2 - The Architecture 
%%-------------------------------------------------------------------
\section{The Architecture} % (fold)
\label{sec:icbd_architecture}

%Topics:
%Introduce the layers
%Draw a diagram 
%Layers are a kind of role, not a single a defined service but a collection 

The iCBD platform is comprised of multiple layers, each comprised of multiple services that take responsibility for a given task. 
We can group this layers in to four major labels as can be seen in the figure~\ref{fig:icbd_layers}


\begin{figure}[htbp]
	\centering
	\includegraphics[height=4in]{placeholder}
	\caption{iCBD Layers View}
	\label{fig:icbd_layers}
\end{figure}


\begin{description}
	\item [Boot Services Layer] Dnsmasq, HTTP, DHCP, TFTP, PXE
	%
	\item [Administration Layer] Virtualisation of Adm machines, creation of icbd boot packages
	%
	\item [Client Support Layer] R/W space, clients home dir, active directory
	%
	\item [Storage Layer] BTRFS, CEPH, NFS, iSCSI 
\end{description}
 
 
Let's view in more detail each one of them.

%%-------------------------------------------------------------------
%%	3.2. - Boot Layer 
%%-------------------------------------------------------------------
\subsection{Boot Layer}
\label{sub:icbd_architecture_boot}

From a end-user perspective the only layer that is visible and interactive is the boot layers.


%%-------------------------------------------------------------------
%%	3.2. - Administration Layer 
%%-------------------------------------------------------------------
\subsection{Administration Layer}
\label{sub:icbd_architecture_adm}


%%-------------------------------------------------------------------
%%	3.2. - Client Layer 
%%-------------------------------------------------------------------
\subsection{Client Layer}
\label{sub:icbd_architecture_client}

%%-------------------------------------------------------------------
%%	3.2. - Storage Layer 
%%-------------------------------------------------------------------
\subsection{Storage Layer}
\label{sub:icbd_architecture_storage}


%%-------------------------------------------------------------------
%%	3.2. - Replication and Caching - The Problem 
%%-------------------------------------------------------------------
\section{Replication and Caching - The Problem}
\label{sec:replication_cache}


\subsection{Motivation and Goals}
\label{sub:motivation_goals}

Cache servers
Para solucionar os problemas antes mencionados que afectam as soluções DaaS, decorrentes da limitada Largura de Banda, latência e jitter apreciáveis das limitações no acesso através da internet aos repositorios de VMs templates.
Para se resolver o problema da limitada largura de banda que aflige as soluções habituais DaaS e para proporcionar a escalabilidade da solução iremos utilizar "servidores de proximidade" ou "Cache Servers", que irão conservar réplicas das VMs templates mantidas nas cloud, no sistema de administração. Será a partir dos cache servers presentes nos segmentos locais das LAN que os PCs clientes irão arrancar. Para isso os cache servers terão de dispor de discos locais (embora fosse possivel ter-se cache servers diskless, ficariam bloqueados caso houvesse uma interrupção no acesso à internet, e é para evitar isso que os disco locais são na pratica necessários.)

%A instalação, configuração e administração dos cache-servers far-se-á também pela instanciação a partir da imagem de uma VM especialmente preparada para o efeito e residente na cloud, no sistema de administração. Os cache-servers arrancam inicialmente pela rede a partir do servidor na cloud e carregam um Linux que vai formatar os discos, neles instalando o conteúdo da própria imagem de VM, terminando com a instalação de um boot loader que irá arrancar o sistema do cache-server após a máquina fazer reboot. Daí em diante o sistema de administração na cloud irá disponibilizar as actualizações que forem necessárias, podendo mesmo forçar a re-instalação total dos cache-servers.

%Note-se que uma vez instalado um cache-server numa LAN, dada a grande fiabilidade que pode ser ainda reforçada através das técnicas habituais em sistemas tolerantes a faltas e de elevada disponibilidade (referir a sugerida pelo Paulo), torna-se viável a utilização de servidores diskless, instanciados a partir de VMs templates configuradas e administradas na cloud e depois replicadas para o cache-server, como nas imagens dos desktops.

%Assim como o Linux dos cache-servers é instalado nos discos locais dos servidores a partir de imagens na cloud, é possível fazer o mesmo com qualquer outra imagem de Linux que se queira. Assim, o administrador de sistema de um cliente poderá configurar outras VMs com o software e os serviços de que necessitar, designar uma máquina física como alvo, e fazer com que a VM seja vertida para os discos da sua máquina física, sendo configurado um boot loader para lhe permitir arrancar com essa configuração.


\subsection{System Overview}
\label{sub:system_overview}

\subsection{Requirements}
\label{sub:requirements}